%%%%%%%%%%%%%%%%%%%%%%%%%%%%%%%%%%%%%%%%%%%%%%%%%%%%%%%%%%%%%%%%%%%%%%%%
%%%%%%%%%%%%%%%%%%%%%% Simple LaTeX CV Template %%%%%%%%%%%%%%%%%%%%%%%%
%%%%%%%%%%%%%%%%%%%%%%%%%%%%%%%%%%%%%%%%%%%%%%%%%%%%%%%%%%%%%%%%%%%%%%%%

%%%%%%%%%%%%%%%%%%%%%%%%%%%%%%%%%%%%%%%%%%%%%%%%%%%%%%%%%%%%%%%%%%%%%%%%
%% NOTE: If you find that it says                                     %%
%%                                                                    %%
%%                           1 of ??                                  %%
%%                                                                    %%
%% at the bottom of your first page, this means that the AUX file     %%
%% was not available when you ran LaTeX on this source. Simply RERUN  %% 
%% LaTeX to get the ``??'' replaced with the number of the last page  %% 
%% of the document. The AUX file will be generated on the first run   %%
%% of LaTeX and used on the second run to fill in all of the          %%
%% references.                                                        %%
%%%%%%%%%%%%%%%%%%%%%%%%%%%%%%%%%%%%%%%%%%%%%%%%%%%%%%%%%%%%%%%%%%%%%%%%

%%%%%%%%%%%%%%%%%%%%%%%%%%%% Document Setup %%%%%%%%%%%%%%%%%%%%%%%%%%%%

% Don't like 10pt? Try 11pt or 12pt
\documentclass[10pt]{article}

% Polish language
\usepackage[utf8]{inputenc}
\usepackage[T1]{fontenc}
% This is a helpful package that puts math inside length specifications
\usepackage{calc}

% For comment
\usepackage{verbatim}

% Layout: Puts the section titles on left side of page
\reversemarginpar

%
%         PAPER SIZE, PAGE NUMBER, AND DOCUMENT LAYOUT NOTES:
%
% The next \usepackage line changes the layout for CV style section
% headings as marginal notes. It also sets up the paper size as either
% letter or A4. By default, letter was used. If A4 paper is desired,
% comment out the letterpaper lines and uncomment the a4paper lines.
%
% As you can see, the margin widths and section title widths can be
% easily adjusted.
%
% ALSO: Notice that the includefoot option can be commented OUT in order
% to put the PAGE NUMBER *IN* the bottom margin. This will make the
% effective text area larger.
%
% IF YOU WISH TO REMOVE THE ``of LASTPAGE'' next to each page number,
% see the note about the +LP and -LP lines below. Comment out the +LP
% and uncomment the -LP.
%
% IF YOU WISH TO REMOVE PAGE NUMBERS, be sure that the includefoot line
% is uncommented and ALSO uncomment the \pagestyle{empty} a few lines
% below.
%

%% Use these lines for letter-sized paper
%\usepackage[paper=letterpaper,
%            %includefoot, % Uncomment to put page number above margin
%            marginparwidth=1.2in,     % Length of section titles
%            marginparsep=.05in,       % Space between titles and text
%            margin=1in,               % 1 inch margins
%            includemp]{geometry}

%% Use these lines for A4-sized paper
\usepackage[paper=a4paper,
            %includefoot, % Uncomment to put page number above margin
            marginparwidth=27.5mm,    % Length of section titles
            marginparsep=1.5mm,       % Space between titles and text
            margin=25mm,              % 25mm margins
            includemp]{geometry}

%% More layout: Get rid of indenting throughout entire document
\setlength{\parindent}{0in}

%% This gives us fun enumeration environments. compactitem will be nice.
\usepackage{paralist}

%% Reference the last page in the page number
%
% NOTE: comment the +LP line and uncomment the -LP line to have page
%       numbers without the ``of ##'' last page reference)
%
% NOTE: uncomment the \pagestyle{empty} line to get rid of all page
%       numbers (make sure includefoot is commented out above)
%
\usepackage{fancyhdr,lastpage}
\pagestyle{fancy}
%\pagestyle{empty}      % Uncomment this to get rid of page numbers
\fancyhf{}\renewcommand{\headrulewidth}{0pt}
\fancyfootoffset{\marginparsep+\marginparwidth}
\newlength{\footpageshift}
\setlength{\footpageshift}
          {0.5\textwidth+0.5\marginparsep+0.5\marginparwidth-2in}
\lfoot{\hspace{\footpageshift}%
       \parbox{4in}{\, \hfill %
%en                   \arabic{page} of \protect\pageref*{LastPage} % +LP
%pl                   \arabic{page} z \protect\pageref*{LastPage}  % +LP
%                    \arabic{page}                               % -LP
                    \hfill \, }}
% Finally, give us PDF bookmarks
\usepackage{color,hyperref}
\definecolor{darkblue}{rgb}{0.0,0.0,0.3}
\hypersetup{colorlinks,breaklinks,
            linkcolor=darkblue,urlcolor=darkblue,
            anchorcolor=darkblue,citecolor=darkblue}

%%%%%%%%%%%%%%%%%%%%%%%% End Document Setup %%%%%%%%%%%%%%%%%%%%%%%%%%%%


%%%%%%%%%%%%%%%%%%%%%%%%%%% Helper Commands %%%%%%%%%%%%%%%%%%%%%%%%%%%%

% The title (name) with a horizontal rule under it
%
% Usage: \makeheading{name}
%
% Place at top of document. It should be the first thing.
\newcommand{\makeheading}[1]%
        {\hspace*{-\marginparsep minus \marginparwidth}%
         \begin{minipage}[t]{\textwidth+\marginparwidth+\marginparsep}%
                {\large \bfseries #1}\hfill {\scriptsize \input{date}}\\[-0.15\baselineskip]%
                 \rule{\columnwidth}{1pt}%
         \end{minipage}
}

% The section headings
%
% Usage: \section{section name}
%
% Follow this section IMMEDIATELY with the first line of the section
% text. Do not put whitespace in between. That is, do this:
%
%       \section{My Information}
%       Here is my information.
%
% and NOT this:
%
%       \section{My Information}
%
%       Here is my information.
%
% Otherwise the top of the section header will not line up with the top
% of the section. Of course, using a single comment character (%) on
% empty lines allows for the function of the first example with the
% readability of the second example.
\renewcommand{\section}[2]%
        {\pagebreak[2]\vspace{1.4\baselineskip}%
         \phantomsection\addcontentsline{toc}{section}{#1}%
         \hspace{0in}%
         \marginpar{
         \raggedright \scshape #1}#2}

\renewcommand{\subsection}[2]%
        {\pagebreak[2]\vspace{0.3\baselineskip}%
         \phantomsection\addcontentsline{toc}{subsection}{#1}%
         \hspace{0in}%
         \marginpar{\scriptsize
         \raggedright \scshape #1}#2}


% An itemize-style list with lots of space between items
\newenvironment{outerlist}[1][\enskip\textbullet]%
        {\begin{itemize}[#1]}{\end{itemize}%
         \vspace{-.6\baselineskip}}

% An environment IDENTICAL to outerlist that has better pre-list spacing
% when used as the first thing in a \section 
\newenvironment{lonelist}[1][\enskip\textbullet]%
        {\vspace{-\baselineskip}\begin{list}{#1}{%
        \setlength{\partopsep}{0pt}%
        \setlength{\topsep}{0pt}}}
        {\end{list}\vspace{-.6\baselineskip}}

% An itemize-style list with little space between items
\newenvironment{innerlist}[1][\enskip\textbullet]%
        {\begin{compactitem}[#1]}{\end{compactitem}}

% To add some paragraph space between lines.
% This also tells LaTeX to preferably break a page on one of these gaps
% if there is a needed pagebreak nearby.
\newcommand{\blankline}{\quad\pagebreak[2]}

%%%%%%%%%%%%%%%%%%%%%%%% End Helper Commands %%%%%%%%%%%%%%%%%%%%%%%%%%%

%%%%%%%%%%%%%%%%%%%%%%%%% Begin CV Document %%%%%%%%%%%%%%%%%%%%%%%%%%%%

\begin{document}
\makeheading{Dawid Ciężarkiewicz}


%en \section{Contact Information}
%pl \section{Dane kontaktowe}
%
% NOTE: Mind where the & separators and \\ breaks are in the following
%       table.
%
% ALSO: \rcollength is the width of the right column of the table 
%       (adjust it to your liking; default is 1.85in).
%
\newlength{\rcollength}\setlength{\rcollength}{2.05in}%
%
\begin{tabular}[t]{@{}p{\textwidth-\rcollength}p{\rcollength}}
%pub \begin{comment}
\input{cv-address1}
%pub \end{comment}
%pub Katowice
%en &\textit{Mobile:}
%pl &\textit{Telefon:}
%pub \begin{comment}
\input{cv-mobile}
%pub \end{comment}
%pub {\it - }
\\
%pub \begin{comment}
\input{cv-address2}
%pub \end{comment}
& \textit{E-mail:}
\href{mailto:dpc@ucore.info}{dpc@ucore.info}\\
%en Poland 
   & \textit{WWW:}
\href{http://dpc.ucore.info/}{http://dpc.ucore.info}\\
\end{tabular}

%en \section{Citizenship}
%en Polish

%en \section{Personal Statement}
%pl \section{Profil Osobisty}
%
%en I'm very flexible, optimistic person with a strong analytic mind.
%en I consider myself to be a self-learner - with inborn predisposition
%en to acquire and utilize new knowledge. 
%en However, I like collaboration and teamwork.
%en Fond of wide area of my experience and interest I'm always seeking
%en opportunities to
%en use and develop my skills. I enjoy challenging work and dealing with
%en cutting-edge technology.
%pl Jestem bardzo elastyczną, optymistycznie nastawioną osobą
%pl obdarzoną bardzo analitycznym umysłem.
%pl Uważam się za samouka -
%pl posiadam wrodzoną predyspozycję do zdobywania wiedzy i wykorzystywania
%pl jej w praktyce.
%pl Lubię współpracę i pracę w grupie.
%pl Świadomy szerokiego obszaru swych zainteresowań i doświadczenia,
%pl zawsze poszukuję okazji do rozwoju i zdobywania
%pl nowych umiejętności.
%pl Cenie sobie wymagające projekty i pracę z najnowszymi technologiami.


%en \section{Professional Interests}
%pl \section{Zainteresowania Zawodowe}
%en %
%pl
%en Unix system software development and administration, embedded devices
%en and low level programming,
%en microkernels,
%en networking, real time communication, cryptography,
%en software design, programming languages design and development
%pl Rozwój oprogramowania systemowego i administracja systemami unixowymi,
%pl oprogramowanie niskopoziomowe i systemy embedded, mikrojądra,
%pl sieci komputerowe, komunikacja czasu rzeczywistego, kryptografia,
%pl projektowanie oprogramowania, projektowanie i rozwój języków programowania


%en \section{Education}
%pl \section{Edukacja}
%
\href{http://www.polsl.pl/}
%en {\textbf{Silesia University of Technology}},
%pl {\textbf{Politechnika Śląska}},
Gliwice
%en, Poland
\begin{outerlist}

\item[]
        \href{http://www.polsl.pl/rau}
%en             {Department of Automatics, Electronics and Computer Science} 
%pl             {Wydział Automatyki, Elektroniki i Informatyki} 
%en        (currently at the last term)
%pl        (obecnie na ostatnim semestrze)
        \begin{innerlist}
%en        \item Thesis Topic: Netfilter
%en          - filtering network traffic in the Linux OS
%pl        \item Temat Pracy: Netfilter
%pl          - filtrowanie ruchu sieciowego w systemie Linux
        \item
%en          Advisor: 
%pl          Prowadzący: 
              %\href{}
%en                   {D. Eng. Jacek Lach}
%pl                   {dr inż. Jacek Lach}
        \end{innerlist}
\end{outerlist}

\begin{comment}
%en \section{Awards}
%pl \section{Wyróżnienia}
%
\href{http://www.oi.edu.pl/}
%pl {Olimpiada Informatyczna organizowana przez Ministerstwo Edukacji Narodowej}
%en {Polish High-School Contents in Computer Science organized by
%en Polish Ministry of Education}

\begin{innerlist}
%en \item {twice promoted to second stage}
%pl \item {dwukrotny awans do drugiego etapu}
: 2002, 2004
\end{innerlist}
\end{comment}

%en \section{Publications}
%pl \section{Publikacje}
%
Dawid Ciężarkiewicz,
\href{http://www.software20.org/pl/linuxplus/issues/9\_2007.html}
                         {Ara – uwierzytelnianie w kolorach},
%pl magazyn Linux+
%en Linux+ magazine
9/2007.

%en \section{Professional Experience}
%pl \section{Doświadczenie Zawodowe}
%
\textbf{\href{http://www.capgemini.pl/}{Capgemini Polska} Sp. z o.o.}, 
Katowice
\begin{outerlist}

\item[] \textit{
%pl Specjalista ds. Wsparcia Technicznego
%en Technical Support Specialist
}%
        \hfill \textbf{
%en March 2008 to present day
%pl Marzec 2008 - obecnie
}
\begin{innerlist}
\item
%pl Zdalne wsparcie techniczne oprogramowania i sprzętu dla anglojęzycznych
%pl klientów firmy SUN Microsystem,
%pl w szczególności: serwerów w architekturach SPARC oraz X64 (klasy
%pl Entry-Level, Midrange, High-End, CoolThreads, stacji roboczych),
%pl systemów telekomunikacyjnych, macierzy dyskowych, bibliotek taśmowych oraz
%pl systemu Solaris
%pl i jego najważniejszych usług (SMF, SVM, VxVM, NFS ...)
%en Remote software and hardware technical support for English speaking
%en SUN Microsystems customers:
%en SPARC and X64 architectures (Entry-Level, Midrange Servers, High-End, Workstations,
%en Telco systems),
%en Storage Arrays, Tape Libraries and Solaris OS and software (SMF, SVM, VxVM,
%en NFS ...)
\end{innerlist}

\end{outerlist}

\blankline

\href{http://asn.pl/}{\textbf{ASN}}, 
ul. Katowicka 115, 41-500 Chorzów
\begin{outerlist}

\item[] \textit{
%en Administrator/Developer
%pl Administrator/Programista
}%
        \hfill \textbf{
%en May 2005 to August 2007
%pl Maj 2005 - Sierpień 2007
}
\begin{innerlist}
\item
%en Monitoring and administration of ISP network for about 300 end-users.
%pl Administracja i monitoring sieci dostępowej dla około 300 klientów końcowych.
\item 
%en Creation and modification of software used for internal usage
%en (system software, Linux kernel, web-based services).
%pl Tworzenie i modyfikacja oprogramowania na potrzeby wewnętrzne
%pl (oprogramowanie systemowe, jądro systemu Linux, serwisy webowe).
\item
%en Creation and developing Linux distribution for network routers and bridges
%pl Tworzenie i rozwijanie dystrybucji systemu Linux przeznaczonej
%pl na routery i mosty sieciowe
- \href{http://lintrack.org}{Lintrack}.
\end{innerlist}

\end{outerlist}

%en \section{Technical \newline
%en Skills}
%pl \section{Umiejętności Techniczne}
%
%en Extensive hardware and software experience in networking and
%en        information technology
%pl Bogate doświadczenie ze sprzętem
%pl oraz oprogramowaniem w dziedzinach związanych z informatyką
%pl i sieciami komputerowymi

\blankline

%en \subsection{Programming languages}
%pl \subsection{Języki programowania}
Assembler
%pl (głównie x86),
%en (mostly x86),
C, C++, D, Java, JavaScript, Pascal, Python, Perl, PHP, Ruby,
%en UNIX shell scripting,
%pl skrypty powłoki UNIX
SQL,
%pl i inne
%en and others

\blankline

%en \subsection{Programming experience}
%pl \subsection{Doświadczenie programistyczne}
%en databases, system and networking software,
%pl bazy danych, oprogramowanie systemowe i sieciowe,
%en web-based applications,
%pl aplikacje webowe,
Unix API,
%en embedded systems,
%pl systemy wbudowane,
%en Linux kernel,
%pl jądro systemu Linux,
XML, XMPP, Qt, UML,
%en working with revision systems
%pl praca z systemami rewizjonowania
(RCS, CVS, SVN, Git
%en and other)
%pl i inne)

\blankline

%en \subsection{Applications}
%pl \subsection{Aplikacje użytkowe}
\TeX{}, \LaTeX{}, B\textsc{ib}\TeX{}, Microsoft Office,
%en and other common productivity packages for
%pl inne znane aplikacje użytkowe dla systemów
Windows,
%en OS X, and
%pl OS X i
 Linux
%en platforms

\blankline

%pl \subsection{Administracja}
%en \subsection{Administration}
 Microsoft Windows XP/2000, Linux, BSD,
        Solaris
%en and other UNIX variants;
%pl i inne warianty systemu UNIX;
%pl sieci TCP/IP;
%en TCP/IP networks;
%pl usługi sieciowe (apache, XMPP, email, SVN i inne)
%en network services (apache, XMPP, email, SVN and other)

%en \subsection{Hardware Support}
%pl \subsection{Diagnoza Sprzętu}
%pl Doświadczenie w diagnozie i administracji serwerami
%pl oraz sprzętem peryferyjnym, ze szczególnym uwzględnieniem
%pl rozwiązań firmy \href{http://sun.com}{SUN Microsystems}
%pl (patrz szkolenia)
%en Experienced in general troubleshooting and administration of
%en servers and peripheral hardware, especially 
%en \href{http://sun.com}{SUN Microsystems} solutions (see courses).

%en \section{Certificates}
%pl \section{Certyfikaty}
Sun Certified System Administrator for Solaris 10 OS

%en examind by Prometric Services
%pl egzamin przeprowadzony przez Prometric Services

%pl Data zdania:
%en Pass date:
6/30/09

\blankline

Sun Global Resolution Certificate

%en certified by Kepner-Tregoe
%pl certyfikat wydany przez Kepner-Tregoe

%pl Data zdania:
%en Pass date:
11/12/2009

%en \section{Courses}
%pl \section{Szkolenia}
%en Numbers of internal SUN Microsystems technical training and courses:
%pl Liczne techniczne szkolenia wewnętrzne SUN Microsystems:

\begin{innerlist}
%en \item general SUN systems training
%pl \item przegląd systemów firmy SUN
%en	\item general hardware diagnosis trainings
%pl \item szkolenia diagnozy urządzeń sprzętowych
%en \item T1000, T2000, T5xx0 systems
%pl \item systemy: T1000, T2000, T5xx0
%en \item SUN Microsystems x64 family
%pl \item rodzina systemów x64 SUN Microsystems
%en \item v240/v440 SPARC family
%pl \item rodzina systemów SPARC: v240/v440
%en \item v490/v890 and v480/v880 SPARC family
%pl \item rodziny systemów SPARC: v490/v890 i v480/v880
\item LW8 Enterprise 2900, SF v1290
%en \item Sun SPARC Enterprise Mx000 (OPL) family
%pl \item rodzina systemów: Sun SPARC Enterprise Mx000 (OPL)
%en \item SUN Blade 6000/8000 family
%pl \item rodzina systemów SUN Blade 6000/8000
%en \item 3xx0 (Minnow), D1000, A1000 arrays families
%pl \item rodziny macierzy: 3xx0 (Minnow), D1000, A1000
%en	\item tape libraries troubleshooting (StorageTek and SUN family)
%pl \item troubleshooting bibliotek taśmowych (StorageTek i SUN)
\item SAN troubleshooting
%en \item general Unix administration
%pl \item administracja systemami z rodziny Unix
\item Solaris 9 and 10 OS
%en \item volume managers: Linux LVM, SVM, VxVM, hardware RAID
%pl \item usługi woluminów logicznych: Linux LVM, SVM, VxVM, RAID sprzętowy
%pl \item troubleshooting macierzy dyskowych z rodziny 3xx0 (Minnow)
\end{innerlist}

\blankline

%en SUN Microsystem softskill internal trainings:
%pl Wewnętrzne szkolenia ,,softskillowe'' dla pracowników firmy
%pl SUN Microsystems:
\begin{innerlist}
\item SUN Global Resolution Training
\item Communication and Influence Workshop
%en \item Customer Care trainings
%pl \item treningi Customer Care
\end{innerlist}


\vfill
%en \section{Legal Note}
%pl \section{Nota Prawna}
%pub \begin{comment}
%en I~hereby authorize you to process my personal data included in my job
%en application for the needs of the recruitment process
%en (in accordance with the Personnel Protection Act 29.08.1997
%en no 133 position 883)
%pl Wyrażam zgodę na przetwarzanie moich danych osobowych w celach
%pl rekrutacyjnych zgodnie z ustawą z dn. 29.08.1997r. o ochronie danych
%pl osobowych, Dz. U. nr 133, poz. 833.
%pub \end{comment}
%pub %en This is public version of Curriculum Vitae for information purposes
%pub %en only.
%pub %en To obtain version for recruitment process, please contact me directly.
%pub %pl Dokument ten jest publiczną wersją mojego Curriculum Vitae przeznaczoną
%pub %pl jedynie do celów informacyjnych. 
%pub %pl W razie zainteresowania wersją przeznaczoną
%pub %pl do celów rekrutacyjnych - proszę o kontakt.
\end{document}

%%%%%%%%%%%%%%%%%%%%%%%%%% End CV Document %%%%%%%%%%%%%%%%%%%%%%%%%%%%%
